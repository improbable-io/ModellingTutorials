\section{Introduction}

Physicists suspect that a large proportion of the mass in the universe (the
mysterious dark matter) is bound up in tiny black holes which were formed
during the big bang. Over time, these black holes have shrunk due to Hawking
radiation but there may still be primordial black holes (PBHs) of up to $10^{26}$g
(around the mass of Mercury) passing through the universe at speeds of around
300 kms-1. Physicists have considered the effect of a smaller PBH striking the
Earth but here, we'll consider a more Improbable World: the effect of a $10^{26}$g
PBH entering our solar system on a possible collision course with Earth.  

Since black holes are...erm...black, we couldn't directly observe the position
and velocity of the PBH so calculating whether it really is on a collision
course (and whether we should continue paying into our salary sacrifice
pension) wouldn't be easy. Luckily, a PBH that massive would potentially
perturb the orbits of the other planets so by watching the wobble of the
planets we can work out the position and velocity of the PBH and so work out if
it is likely to hit the Earth. 

\subsection{Modelling goals}

\begin{itemize}
\item Create a physically realistic model of our solar system
\item Use observations of the position of Mercury to calculate the position and
velocity of a passing PBH and work out if it is on a collision course with
Earth
\end{itemize}

\subsection{Take-home insights}

\color{red} TODO: based upon the position of this tutorial in the series and the 
preceding lessons. \color{black}
