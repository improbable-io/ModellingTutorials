\section{System identification and decomposition}

Thankfully, we all studied and remember A-level physics (!) and therefore
have a good amount of domain knowledge already at our fingertips. In this case, 
the system boundaries are (unusually) well defined -- we wouldn't expect any
interaction between objects inside and outside the solar system - so we can 
limit our modelling to that extent. The main system components are the planets,
the PBH and the Sun - these can be grouped as 'massive bodies' (i.e. bodies that
have mass). My intuition, at this stage, is that satellites are unlikely to play
a big role in the problem (Mercury has no known satellites -- and by the time
the PBH is close to Earth, it's too late for the Moon to have much impact). Of
course a rigorous modelling approach would support this assertion with appropriate
computation.

The important states of our massive bodies are their location, $\vec{x}(t)$, and 
velocity, $\vec{u}(t)$. Massive bodies interact via the gravitational force their
mass exerts, we recall (!) that the gravitational force, $\vec{F_{g}}$, between
two bodies with masses $m_1$ and $m_2$ separated by $\vec{r}$ is
\begin{equation}
\vec{F_{g}} = - \frac{G m_1 m_2 \vec{r}}{|\vec{r}|^3}.
\end{equation}
We further recall (!) that an unbalanced force acting on a body will induce an
acceleration
\begin{equation}
\vec{F_{g}} = m \vec{a},
\end{equation}
and that this acceleration can be used to update the velocity and position of
the body. Thus, $\vec{a}(t)$ is an intermediate property of our massive bodies 
that is important to keep track of, and the seperation distance between pairs
of bodies is an important factor controlling the extent of their interaction.

In this case, timestepping would seem a sensible temporal scheme, we define a
timestep length, $\Delta t$ (often just called $dt$), at each iteration we'll
step forward in time, compute the resultant gravitational forces acting on
each body, compute the acceleration induced on each body, update the velocity
of each body and finally compute the new position of each body. Note that this
temporal discretisation induces an error (a `discretisation error' no less) --
we are assuming that the acceleration of a body is constant for the duration of
the timestep, where patently this is false. We can ultimately measure the impact
of this error by exploring the effect of adjusting $dt$ in our final model.
