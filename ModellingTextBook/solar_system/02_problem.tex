\section{Problem formulation}

\subsection{What is the problem}
The core problem is: Will the Earth be hit by a primordial black hole? An 
ancillary problem is that PBH's cannot be directly observed, and as such their
position must be inferred by their effects on the massive bodies they may effect.

\subsection{Hypothesised approach}
The mechanics of orbiting bodies are well understood, and can be modelled
accurately with relative ease. In this scenario we are astronomers who want to
build a model of the solar system which is able to detect planetary wobble due
to a primordial black hole (PBH). So the system we build must be able to
compare its own output to observations made in the real world and detect
discrepancies due to planetary wobble. Given this wobble we need to be able to
estimate the location of the PBH at a given time, then add this to the model
and run it forwards into the future to determine the likely trajectory of the
PBH (given the errors in estimating its location in the first place) and
ultimately compute a probability of collision with earth.

\begin{itemize}
\item Build an accurate model of the dynamics of the solar system in `normal'
conditions.
\item Monitor the difference between the observed  position (i.e. the position
as observed by real-world telescopes) of Mercury and its position in the model.
\item Estimate the  position of the PBH and bound the uncertainty attached to
that estimation.
\item Estimate the trajectory of the PBH and the probability that it will
collide with Earth.
\end{itemize}

\subsection{Problem owner and additional stakeholders}
Since this is a purely technical system, we do not need to consider whose
problem we are addressing, and the other stakeholders and actors whose
behaviour and motivations would be an important dynamic in a social or
socio-technical system.

\subsection{Our role}
Our role, as modellers, is to extract knowledge from domain experts to build
a model which addresses the problem as formulated in bullet points above. Perhaps
most importantly, it is to communicate the assumptions under which the results
of the model will hold alongside those results.
