\section{Structuring models for the ScienceSDK}

Given the intent of this booklet as an introduction to modelling on the
ScienceSDK, let's look more closely at how to formalise our model to best suit
the structure of that tool. See chapter XXX for an introduction to the
ScienceSDK. One of the core concepts of the ScienceSDK is to present to the
modeller a familiar object orientated programming paradigm (rather than the
message passing paradigm of SpatialOS which is likely to be less familiar to
the modelling community).  Consequently, the key programming unit in ScienceSDK
is the \textit{Migratable}.  A \textit{Migratable} is an object which is
capable of being moved around in entirety to different processors.
\textit{Migratable} objects are constructed with corresponding
\textit{Reference} objects which are constructed for the user by the
ScienceSDK. \textit{Migratables} refer to one another indirectly via the
\textit{Reference}s, which provide pointers to the objects themselves, on
whichever processor they may currently be residing. A migratable is
consequently the smallest unit of code in the ScienceSDK paradigm.

So how can we structure concepts and models around this paradigm? Well the
obvious mapping is the tangible agents and objects to migratables, with more
abstract objects -- for example observation operators -- also treated
similarly.  The spatial paradigm is an intrinsic characteristic of the
underlying SpatialOS treatment of objects (spatial location is a key
characteristic used for load-balancing) but ScienceSDK abstracts this away from
the user -- where it can be appropriately applied to a migratable it may be
defined by the modeller, where inappropriate it shouldn't be defined, allowing
the ScienceSDK to load-balance more effectively.



