\section{Glossary of terms}

We all come from different backgrounds -- and our core tech comes from the
games-sphere -- and have picked up our modelling jargon from a diverse range of
sources. It would be helpful if, within the division, we could develop a
unified language with which to approach problems.

I think this would also be helpful for the wider company to interact with what
we do on this side of the wall.

\color{red} Tidy up the above \color{black}

\subsection{Model vs. Simulation}

\noindent \textbf{Model:} A representation, or network of interconnected representations,
of a concept/process or multiple concepts/processes.
\textit{Modelling a process implies you have discarded detail which is
extraneous to the problem under-consideration and constructed a (computational)
representation at an appropriate level of abstraction for the problem.}

\noindent \textbf{Simulation:} a model which has some proven legitimacy within some concretely
defined scope to claim to represent (i.e. simulate) some real concept or
process (ideally with some quantification of error and uncertainty).
\textit{Simulating a process implies you are augmenting your modelling effort with a
good standard of calibration and validation. For a simulation, the level of
modelling abstraction will be determined by both the problem \& the available
data.}

\subsection{Time}

\noindent \textbf{Real-time:} Time is represented in the model as it is in the real-world
(albeit potentially sped-up) - consequently the progress of the model is
dependent upon time.  \textit{Modelling a system which is active Mon, Tues \& Thurs, but
dormant on Wed - would mean the hardware sits idle for the Wed part of the
simulation.}

\noindent \textbf{Simulated time:} Time in the model is dependent upon program progress. Methods
such as discrete-time and constant step duration iteration are examples of the
use of simulated time.  \textit{In the above example, we can use simulated time to
ensure the hardware will skip over the idle Wed and jump straight to Thursday’s
work. Thereby we can solve problems as fast as the computer can compute,
independently of problem’s relationship to real-world time.}

\subsection{Types of systems}

\noindent \textbf{Complex system:} A system which displays emergence -- features
at the collective-scale which develop from the interactions of a large number
of individuals at the unit-scale.

\noindent \textbf{Complex adaptive system:} A complex system in which the individuals at the
unit-scale will adjust their behaviour dependent upon the features that develop
at the collective scale. \textit{Economics is the canonical complex adaptive system -
the actors who comprise an economy will modify their behaviour in response to
the emergent features of it.}

\noindent \textbf{Technical:} Systems whose scope is non-behavioural - i.e. limited to purely
scientific or technical processes.

\noindent \textbf{Social:} Systems whose scope is (almost) entirely behavioural.

\noindent \textbf{Socio-technical:} Systems in which the phenomena of interest involve the
interaction of behavioural and technical behaviours / processes.

\subsection{Types of model}

\noindent \textbf{Discrete event:} A model which uses a discrete event treatment of
time. Processors work through their task immediately, one after another --
meaning that one processor may be ahead or behind another -- there are various
schemes which deal with this disparity in ways that may be more or less
suitable for a given problem.

\noindent \textbf{Discrete time:} A time-stepping scheme whereby each iteration covers a small
change in time and for each iteration there is a set of tasks that must be
worked through.

\noindent \textbf{Live-streaming:} A model that is driven by real-world data - literally just a
visualisation of the real world phenomena.  \textit{E.g. the map on the uber app - the
position of the icons representing the cars is determined by the live-streamed
position of the cars in the real world -- there is no computation or logic used
in their placement.}

\noindent \textbf{Live-action:} A model which (likely) runs in real-time and allows users to
play-along to evaluate the consequences of a given action. \textit{In non-deterministic
models this evaluation will be crude (as repeating the computational experiment
requires the ‘player’ to repeat the action over and over).  SpatialOS (as a
games platform) is clearly well suited to these. Note that validation and
calibration is very difficult - one cannot be sure that the ‘player’ will
behave in a manner that has been anticipated and for which the model has been
validated.}

\noindent \textbf{Speculative:} A model which is, for all intents and purposes, un-validatable
either because the data isn’t available or because the scope of the problem is
so wide that the range of possible outcomes is as big as to be impossible to
prove valid. \textit{This is not to cast aspersions on such a model’s
usefulness - indeed the competition to such a model may be a domain expert sat
in a room saying ‘I reckon’ in which case such models will almost certainly
provide insight.}

\subsection{Modelling jargon} 

\noindent \textbf{State:} The current condition of the model, defined through the
values of each of the model parameters. A capture of the model’s state should
behave like a checkpoint and contain all the requisite information to restart
the model from the captured point.  \textit{In Spatial, ‘snapshots’ are used to capture
the model state.}

\noindent \textbf{Determinism:} A given set of conditions will develop in the exact same way every
time the model is run.  A deterministic model will produce exactly the same
answer every time it is run from the same starting state.

\noindent \textbf{Stochasticity:} Randomness - generally a stochastic decision will be made by
taking a random sample from a probability distribution with which we have
represented a behaviour.  \textit{A stochastic model will produce a different solutions
when run from the same initial state. The modeller must take steps to ensure a
given solution returned by the model is in some way representative  - and not
an outlier. For example, a random sample of me asking what you had for
breakfast yesterday will not necessarily equip me to make sweeping statements
about your general breakfast habits - though of course it’s better than
nothing!}

\noindent \textbf{Fidelity:} Can be thought of a bit like resolution - how the unit-scale in the
model compares to the way the problem plays out in the real-world.  \textit{We may
model a pile of sand as a single object - a pile (low fidelity), or we may
model each grain of sand (high fidelity). We could further model the pile at an
atomic-level or smaller, the chosen model fidelity must be motivated by the
problem being solved (and the available validation data in the case of
simulations).}

\noindent \textbf{Abstraction:} The process of suppressing more complex details in order to focus
on the mechanisms that are important to tackling the problem as posed.

\noindent \textbf{Markov:} Future states may be dependent on the current state but are not
dependent on previous ones.

