\section{An approach to modelling - Cheatsheet}

\subsection{The ten steps to modelling}

Following these ten steps will ensure you consider everything you need to
consider.  In the examples presented in this booklet we will step through them
explicitly -- clearly the experienced modeller will do a lot of these steps
implicitly. However, here at Improbable we have a specific product (SpatialOS)
so there may be a tendency to put the cart before the horse (i.e. start with a
modelling approach -- spatial -- and working back to a problem). Of course, an
element of this is necessary to help our clients understand how we can help
them, but it is worthwhile always having a structured approach to modelling in
one's back pocket.

\subsubsection{1. Problem formulation}
\begin{itemize}
\item What is the problem
\item What is the hypothesis
\item Who owns the problem
\item Who are the secondary, tertiary... stakeholders
\item What is our role, what does success look like?
\end{itemize}

\subsubsection{2. System identification and decomposition}
\begin{itemize}
\item Inventory
\begin{itemize}
\item Important (and relevant) concepts
\item Actors and objects (and how these relate to data we can observe)
\item Key measurements and performance indicators
\item Behaviours
\item States and properties
\end{itemize}
\item Structure
\begin{itemize}
\item States and behaviours of system components
\item Interactions between different system components
\item How behaviour at boundaries will be dealt with
\item How behaviours may be grouped into iterations, timesteps etc.
\end{itemize}
\item Consider data availability -- this will determine
\begin{itemize}
\item Model fidelity
\item Flexibility of future model for calibration and validation
\item How model is structured to answer the problem
\end{itemize}
\end{itemize}

\subsubsection{3. Concept and model formalisation}
\begin{itemize}
\item Identify the class of model (governed by the problem and available data)
\item Research how others have tackled similar problems previously (don't reinvent 
the wheel)
\item Concretely define concepts in a rigorous framework
\item Indentify appropriate data structures to codify concepts
\item Define an ontology
\item Model and parametrise behaviours and interactions formally (using maths / logic)
\item Determine the model narrative using pseudo code (structure dependent upon the chosen time scheme)
\item Don't forget input/output mechanisms often via observation operators
\end{itemize}


\subsubsection{4. Boundary and initial conditions}
\begin{itemize}
\item Models generally have dependecy upon previous state -- good choice of initial state is vital
\item Consider likely spin-up duration and behaviour
\item Think carefully about how to model external influence at system boudnaries
\item Be explicit about these treatments thay will likely strongly impact model 
results
\end{itemize}

\subsubsection{5. Software implementation}

\subsubsection{6. Model verification}

\subsubsection{7. Experimentation}

\subsubsection{8. Data analysis}

\subsubsection{9. Model validation}

\subsubsection{10. Model use}
