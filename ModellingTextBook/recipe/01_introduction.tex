\section{Introduction}

Models are simplifications of real-world processes that enable us to make
predictions about phenomena. The process of building a model involves using
observational data to link (or ‘map’) inputs to outputs. Scientists and
designers can then use this map to predict the system’s response to a given
input without having to rerun the experiment. These uses may involve the future
of the planet, as in the case of climate change research, or the safety of
hundreds of people, as in the case of building and aerospace design.

As the power and availability of computational resources increase with time,
computational modelling has become a key cornerstone in the development of
modern scientific understanding, where it is used in concert with laboratory
and field testing. While real-world data can give isolated snapshots of
phenomena and processes, it is the job of the computational model to firstly
recreate this snap-shot (validation) then ‘fill in the blanks’ to give a more
complete picture as to what activity is occurring.
