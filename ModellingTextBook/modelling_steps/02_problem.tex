\section{Problem formulation}

Modelling is a means to an ends -- rather than an end in and of itself. The goal
of most models is to provide insight into the possible nature of a real-world
system. They allow the modeller run `in silico' experiments that would be
difficult or expensive to run `in vitro'.

Models are tools to help us understand the dynamics of systems, explore possible
futures or experiment with scenarios and decisions to determine whether they
yield desirable or detrimental states.

In pure science, addressing the lack of insight is the goal, in technical
systems -- engineering applications -- we use the insight to inform design
decisions and in social applications we use the insight to inform policy
design.

If modelling exists to address deficiencies in insight, then the process is
necessarily goal oriented, and well formulated problems are more likely to yield
useful insights. Thus, the definition of well formulated problem lies at the heart
of a successful modelling strategy.

