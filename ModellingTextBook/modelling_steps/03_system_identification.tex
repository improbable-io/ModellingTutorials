\section{System identification and decomposition}

The next step in approaching the model is to formulate a generativist 
description of the problem. This means the system must be identified and bounded,
then decomposed into the constituent components that we will need to represent in
order to simulate the behaviour of interest of the system under anaylsis.

At this stage we're not looking to formalise the representation of the system
or its constituent components -- that is the next step -- resist the temptation
yet to think about \emph{how} you will model things and focus instead on 
\emph{what} you will need to model in order to tackle the problem as it was
defined in the previous step. Hopefully this illustrates the importance of
the previous step, if the problem identified is "how high will this ball bounce"
we can, at this stage decide that we need not capture in the model that the ball
happens to be red.

It is at this stage that we, as modellers, need to seek expert input. It is not
reasonable to expect us to know everything (or indeed anything) about every
problem with which we are presented. Of course our modeller's intuition will
invariably feed in at this stage, but this should be augmented and verified
through conversations with domain specialists and other modellers (who may
have differing backgrounds and thus a different intuition) and reading of 
academic papers, websites etc. etc. Remember that there is nothing new under the
sun, and the overwhelming liklihood is that someone will have tackled this, or
a similar problem, before -- take every opportunity to learn from their conclusions..

This step is concerned with \textbf{inventory} and \textbf{structure}. The former
is concerned with identifying the boundaries and components of the system and their 
important states etc. 

\textbf{Inventory} - identify the following:
\begin{itemize}
\item Important (and relevant) concepts
\item Actors and objects (and how these relate to data we can observe)
\item Key measurements and performance indicators (linked to the problem!)
\item Behaviours
\item States and properties
\end{itemize}

The structuring phase is then concerned with how the components interact with 
one another and the system boundary, and how we may most appropriately handle
spatio-temporal discretisation.

\textbf{Structure} - consider the following:
\begin{itemize}
\item States and behaviours of system components
\item Interactions between different system components
\item How behaviour at boundaries will be dealt with
\item How behaviours may be grouped into iterations, timesteps etc.
\end{itemize}

Much of the goal of this step is to consider the appropriate model fidelity.
This encapsulates a host of factors that must be appraised, and this process
will, to some extent bleed through into the next modelling step.

In addition it is worth, at this stage, thinking about the data you have (or
are likely to be able to get), is the modelling fidelity appropriate to the
fidelity of the data? How will the model be structured to facilitate validation?
What sources of uncertainty are likely to exist and how will we minimise, quantify
and communicate it?
